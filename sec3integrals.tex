\section{The Integral}

In the following, we assume $(\Omega,\mathcal{A},\mu)$ to be a measure space. We denote by $\mathcal{E}$ the vector space of simple functions on $(\Omega,\mathcal{A})$ and by $\mathbb{E}^+=\{f\in E:f\geq 0\}$ the cone of nonnegative simple functions. If
$$f=\sum_{i=1}^n \alpha_i\indic_{A_i}$$
for some $n\in\mathbb{N}$, mutually disjoint sets $A_1,\ldots,A_n\in\mathcal{A}$ and $\alpha_1,\ldots,\alpha_n\in(0,\infty)$ then this representation is said to be a \textit{normal representation of $f$}.

\subsection{Set Up to Define the Integral}

\begin{lemma}
    Let $f=\sum_{i=1}^m\alpha_i\indic_{A_i}$ and $f=\sum_{j=1}^n\beta_j\indic_{B_j}$ be two normal representations of $f\in\mathbb{E}^+$. Then
    $$\sum_{i=1}^m\alpha_i\mu(A_i)=\sum_{j=1}^n\beta_j\mu(B_j)$$
\end{lemma}
\begin{proof}
    Clearly, if $\alpha_i\neq 0$ for some $i$, then $A_i\subseteq\bigcup_{j=1}^n B_j$. A similar result holds for $B_j$. Thus,
    $$\sum_{i=1}^m\alpha_i\mu(A_i) = \sum_{i=1}^m\sum_{j=1}^n\alpha_i\mu(A_i\cap B_j).$$
    If $\mu(A_i\cap B_j)\neq 0$, then $f(\omega)=\alpha_i=\beta_j$ for any $\omega\in A_i\cap B_j$. Therefore,
    $$\sum_{i=1}^m\sum_{j=1}^n\alpha_i\mu(A_i\cap B_j) = \sum_{j=1}^n\sum_{i=1}^m\beta_j\mu(A_i\cap B_j) = \sum_{j=1}^n\beta_j\mu(B_j).$$
\end{proof}

\begin{definition}
    Define $I:\mathbb{E}^+\to[0,\infty]$ by
    $$I(f) = \sum_{i=1}^m\alpha_i\mu(A_i)$$
    if $f$ has the normal representation $f=\sum_{i=1}^m\alpha_i\indic_{A_i}$.
\end{definition}
The above definition makes sense due to the previous lemma.

\begin{lemma}
\label{properties Of I for simple}
    Let $f,g\in\mathbb{E}^+$ and $\alpha\geq 0$. Then
    \begin{enumerate}[(a)]
        \item $I(\alpha f)=\alpha I(f)$,
        \item $I(f+g)=I(f)+I(g)$, and
        \item If $f\leq g$, then $I(f)\leq I(g)$.
    \end{enumerate}
\end{lemma}
We leave the proof of this theorem as an exercise to the reader.

\begin{definition}[Integral]
    If $f:\Omega\to[0,\infty]$ is measurable, then we define the \textit{integral} of $f$ with respect to $\mu$ by
    $$\int f\d{\mu} = \sup\left\{I(g):g\in\mathbb{E}^+, g\leq f\right\}.$$
\end{definition}

Note that by \cref{properties Of I for simple}(iii), $I(f)=\int f\d{\mu}$ for any $f\in\mathbb{E}^+$. That is, the integral is an extension of $I$ from $\mathbb{E}^+$ to the set of non-negative measurable functions. We expand this to measurable functions in general in the next subsection.

\vspace{2mm}
Let $f,g:\Omega\to\overline{\mathbb{R}}$. Similar to how we write $f\leq g$ if $f(\omega)\leq g(\omega)$ for all $\omega\in\Omega$, we write $f\leq g$ almost everywhere if there exists some set $N\in\mathcal{A}$ such that $\mu(N)=0$ and $f(\omega)\leq g(\omega)$ for all $\omega\in\Omega\setminus N$.

\begin{theorem}
\label{properties of integral 2}
    Let $f,g,f_1,f_2,\ldots$ be measurable maps $\Omega\to[0,\infty]$. Then
    \begin{enumerate}[(a)]
        \item If $f\leq g$, then $\int f\d{\mu}\leq \int g\d{\mu}$.
        \item If $f_n \uparrow f$, then $\int f_n\d{\mu} \uparrow \int f\d{\mu}$.
        \item If $\alpha,\beta\in[0,\infty]$, then
        $$\int(\alpha f+\beta g)\d{\mu}=\alpha\int f\d{\mu} + \beta\int g\d{\mu}$$
        where we take $\infty\cdot 0=0$.
    \end{enumerate}
\end{theorem}
\begin{proof}
    ~
    \begin{enumerate}[(a)]
        \item This is obvious from the definition of the integral.
        
        \item By the definition of the integral,
        
        $$\lim_{n\to\infty}\int f_n\d{\mu}=\sup_{n\in\mathbb{N}}\int f_n\d{\mu}\leq \int f\d{\mu}$$
        
        We must now show that $\int f\d{\mu}\leq \sup_{n\in\mathbb{N}}\int f_n\d{\mu}$. Let $g\in\mathbb{E}^+$ with $g\leq f$. It is enough to show that $\sup_{n\in\mathbb{N}}\int f_n\d{\mu}\geq \int g\d{\mu}$.
        Fix some $t\in(0,1)$. For each $n\in\mathbb{N}$, define
        
        $$A_n = \{\omega\in\Omega:f_n(x)\geq tg(x)\}.$$
        
        Note that each $A_n$ is measurable (Why?) and that $A_i\subset A_{i+1}$ for each $i\in\mathbb{N}$.
        
        We first claim that $\bigcup_{i=1}^\infty A_i = \Omega$. We prove this as follows. For any $\omega\in\Omega$,
        \begin{itemize}
            \item If $f(\omega)\leq tg(\omega)$, then $f(\omega)=0$ and $\omega\in A_n$ for every $n\in\mathbb{N}$.
            
            \item If $f(\omega)> tg(\omega)$, then there exists some $n\in\mathbb{N}$ such that $f_n(\omega)>tg(\omega)$. It follows that $\omega\in A_n$.
        \end{itemize}
        Therefore, $\Omega\subseteq\bigcup_{i=1}^\infty A_i$. Since the reverse inclusion is obviously true, we have $\bigcup_{i=1}^n A_i = \Omega$.
        
        Now,
        $$\int f_n\d{\mu} \geq \int tg\indic_{A_n}\d{\mu}$$
        Taking the limit as $n\to\infty$ and $t\to 1$, we have
        $$\lim_{n\to\infty}\int f_n\d{\mu}\geq \int g\d{\mu}.$$
        This completes the proof.
        
        \item By \cref{measurable function simple sequence}, there exist sequences $(f_n)_{n\in\mathbb{N}}$ and $(g_n)_{n\in\mathbb{N}}$ in $\mathbb{E}^+$ such that $f_n\uparrow f$ and $g_n\uparrow g$. Then by \cref{properties Of I for simple} and (ii),
        \begin{align*}
            \int (\alpha f+\beta g)\d{\mu} &= \lim_{n\to\infty} \int(\alpha f+\beta g)\d{\mu} \\
            &= \alpha \lim_{n\to\infty} \int f\d{\mu} + \beta \lim_{n\to\infty} \int g\d{\mu} = \alpha\int f\d{\mu} + \beta\int g\d{\mu}.
        \end{align*}
    \end{enumerate}
\end{proof}

% \begin{theorem}
% \label{ae integral equality}
%     Let $f,g$ be measurable maps $\Omega\to[0,\infty]$. If $f\leq g$ almost everywhere, Then $\int f\d\mu\leq \int g\d\mu$. In particular, if $f=g$ almost everywhere, then $\int f\d\mu=\int g\d\mu$.
% \end{theorem}
% \begin{proof}
    
% \end{proof}

\subsection{The Integral and some Properties}

We have now introduced enough to define the integral for measurable functions in general.

\begin{definition}[Integrals of Measurable Functions]
\label{defIntegral}
    Let $f:\Omega\to\overline{\mathbb{R}}$ be measurable. We call $f$ \textit{$\mu$-integrable} if $\int |f|\d{\mu}<\infty$ and write
    $$\mathcal{L}^1(\mu)=\mathcal{L}^1(\Omega,\mathcal{A},\mu)=\left\{f:\Omega\to\overline{\mathbb{R}}:f\text{ is }\mu\text{-integrable}\right\}.$$
    For $f\in\mathcal{L}^1(\mu)$, we define the integral of $f$ with respect to $\mu$ by
    $$\int f(\omega)\mu(\d{\omega})=\int f\d{\mu}=\int f^+\d{\mu} - \int f^-\d{\mu}.$$
    For $A\in\mathcal{A}$, we define
    $$\int_A f\d{\mu} = \int (f\indic_A)\d{\mu}.$$
\end{definition}

\begin{theorem}
\label{properties of integral ae 1}
    Let $f:\Omega\to[0,\infty]$ be measurable.
    \begin{enumerate}[(a)]
        \item $f=0$ almost everywhere if and only if $\int f\d{\mu} = 0$.
        \item If $\int f\d{\mu}<\infty$, then $f<\infty$ almost everywhere.
    \end{enumerate}
\end{theorem}
\begin{proof}
    ~
    \begin{enumerate}[(a)]
        \item Let us first prove the forward implication. Let $P=\{\omega\in\Omega:f(\omega)>0\}$. Then $f\leq \infty\cdot\indic_{P}$. As $n\indic_{N}\uparrow\infty\indic_{N}$, by \cref{properties of integral 2},
        $$0\leq\int f\d{\mu}\leq\lim_{n\to\infty}\int n\indic_{N}\d{\mu}=0.$$
        
        For the backward implication, let $A_n=\{\omega\in\Omega:f(\omega)\geq 1/n\}$. Then $A_n\uparrow P$ and for any $n\in\mathbb{N}$,
        $$0=\int f\d{\mu} \geq \int \frac{1}{n}\indic_{A_n} = \frac{\mu(A_n)}{n}.$$
        This implies that $\mu(A_n)=0$ for any $n\in\mathbb{N}$ and therefore, $\mu(P)=0$.
        
        \item Let $A=\{\omega\in\Omega:f(\omega)=\infty\}$. Then for any $n\in\mathbb{N}$,
        $$\mu(A)=\int \indic_{A}\d{\mu}\leq \frac{1}{n}\int f\indic_{f\geq n}\leq \frac{1}{n} \int f\d{\mu}.$$
        Taking the limit as $n\to\infty$, we get $\mu(A)=0$.
    \end{enumerate}
\end{proof}

We now expand some of the properties that we proved earlier for non-negative measurable functions to measurable functions in general.

\begin{theorem}
\label{properties of integral final}
    Let $f,g\in\mathcal{L}^1(\mu)$.
    \begin{enumerate}[(a)]
        \item (Monotonicity). If $f\leq g$ almost everywhere, then $\int f\d\mu\leq \int g\d\mu$. In particular, if $f=g$ almost everywhere, then $\int f\d{\mu}= \int g\d{\mu}$.
        \item (Triangle Inequality). $|\int f\d\mu|\leq \int |f|\d\mu$.
        \item (Linearity). If $\alpha,\beta\in\mathbb{R}$, then $\alpha f+\beta g\in\mathcal{L}^1(\mu)$ and
        $$\int \alpha f+ \beta g\d\mu = \alpha\int f\d\mu + \beta\int g\d\mu.$$
    \end{enumerate}
\end{theorem}
\begin{proof}
    ~
    \begin{enumerate}[(a)]
        \item Since $f\leq g$ almost everywhere, $f^+\leq g^+$ and $f^-\geq g^-$ almost everywhere. It is enough to show that $\int f^+\d\mu\leq \int g^+\d\mu$ and $\int f^-\d\mu\geq \int g^-\d\mu$. Let us prove the former. The latter can similarly be shown.
        
        Let $h=g^+-f^+$. As $h^-=0$ almost everywhere, by \cref{properties of integral ae 1}, $\int h\d\mu=\int h^+\d\mu\geq 0$. The result follows.
        
        \item We have
        \begin{align*}
            \left|\int f\d\mu\right| &= \left|\int f^+\d\mu - \int f^-\d\mu\right| \\
            &\leq \int (f^+ + f^-)\d\mu \\ &= \int |f|\d\mu.
        \end{align*}
        
        \item To show linearity, it suffices to show that for any $\alpha\in[0,\infty)$,
        \begin{itemize}
            \item $\int (f+g)\d\mu = \int f\d\mu + \int g\d\mu$,
            \item $\int \alpha f\d\mu = \alpha\int f\d\mu$, and
            \item $\int (-f)\d\mu = -\int f\d\mu$.
        \end{itemize}
        These are easily shown by splitting each function $f$ into $f^+$ and $f^-$ and using \cref{properties of integral 2} wherever necessary.
    \end{enumerate}
\end{proof}

\begin{definition}
    Let $f:\Omega\to[0,\infty)$ be measurable. Define the measure $\nu$ by
    $$\nu(A)=\int (\indic_A f)\d\mu\text{ for }A\in\mathcal{A}.$$
    Then $\nu$ is said to have \textit{density $f$} with respect to $\mu$. We also denote $\nu$ by $f\mu$.
\end{definition}

In showing that $\nu$ is a measure using \cref{tripledoubleEquivalence}, finite additivity follows from finite additivity of the integral and lower semicontinuity follows from \cref{Monotone Convergence Theorem}.

\begin{theorem}
    Let $f:\Omega\to[0,\infty)$ and $g:\Omega\to\overline{\mathbb{R}}$ be measurable. Then $g\in\mathcal{L}^1(f\mu)$ if and only if $(gf)\in\mathcal{L}^1(\mu)$. In this case, $$\int g\d{(f\mu)} = \int (gf)\d\mu.$$
\end{theorem}
We omit the proof of the above. It may be shown by first assuming $g$ to be an indicator function, then extending this to simple functions, non-negative measurable functions and measurable functions.

\begin{definition}
    Let $f:\Omega\to\overline{\mathbb{R}}$ be measurable and $p\in[1,\infty)$. Define
    $$\norm{f}_p=\left(\int |f|^p\d\mu\right)^{1/p}\text{ and }$$
    $$\norm{f}_\infty=\inf \{k\geq 0:\mu(\{|f|\geq k\})=0\}.$$
    Further, for any $p\in[0,\infty]$, we define the vector space
    $$\mathcal{L}^p(\mu) = \{f:\Omega\to\overline{\mathbb{R}} : f\text{ is measurable and }\norm{f}_p<\infty\}.$$
\end{definition}

\begin{theorem}
    The map $\norm{\cdot}_1$ is a seminorm on $\mathcal{L}^1(\mu)$, that is, for any $f,g\in\mathcal{L}^1(\mu)$ and $\alpha\in\mathbb{R}$,
    \begin{align*}
        \norm{\alpha f}_1 &= |\alpha|\cdot\norm{f}_1 \\
        \norm{f+g}_1 &\leq \norm{f}_1 + \norm{g}_1 \\
        \norm{f}_1 &\geq 0\text{ with equality if and only if }f=0\text{ almost everywhere.}
    \end{align*}
\end{theorem}
\begin{proof}
    The first and third (in)equalities follow from \cref{properties of integral final}(c) and \cref{properties of integral ae 1}(a) respectively. The second inequality follows from the fact that $|f+g|\leq |f|+|g|$. We leave the details of the proof to the reader.
\end{proof}

In fact, $\norm{\cdot}_p$ is a seminorm on $\mathcal{L}^p(\mu)$ for any $p\in[1,\infty]$. The proofs of the first and third (in)equalities are similarly straightforward. The proof of the second however requires Minkowski's inequality. %MINKOWSKI INEQUALITY!!!!!!!!!!!!!!!!!!!!!!!!!!!!!!!!!!!!!!!!!!!!!!!!!!!!!!!!!!!!!!!!!!!!!!!!!!!!!!!!!!!!!!!!!!!!!!!!!!!!!!!!!!!!!!!!!!!!!!!!!!!!!!!!!!!!!!!!!!!!!!!!!!!!!!!!!!!!!!!!!!!!!!!!!!!!!!!!!!!!!!!!!!!!!!!!!!!!!!!!!!!!!!!!!!!!!!!!!!!!!!

\begin{theorem}
    Let $\mu(\Omega)<\infty$ and $1\leq p'\leq p\leq \infty$. Then $\mathcal{L}^{p}(\mu)\subseteq\mathcal{L}^{p'}(\mu)$ and further, the canonical inclusion $\mathcal{L}^{p}(\mu)\hookrightarrow\mathcal{L}^{p'}(\mu)$ given by $f\mapsto f$ is continuous.
\end{theorem}
\begin{proof}
    Let us first take the case where $p=\infty$. For any $f\in\mathcal{L}^\infty(\mu)$, since $|f|\leq \norm{f}_\infty$ almost everywhere,
    $$\int |f|^{p'}\d\mu \leq \int \norm{f}_\infty^{p'} \d\mu = \mu(\Omega)\norm{f}_\infty^{p'} < \infty.$$
    It follows that for any $f,g\in\mathcal{L}^\infty(\mu)$,
    $$\norm{f-g}_{p'} \leq (\mu(\Omega))^{1/p'}\norm{f-g}_\infty$$
    and so the inclusion map is continuous.
    
    Let us next take the case where $p$ is finite. Then for any $f\in\mathcal{L}^{p}(\mu)$ since $|f|^{p'}\leq 1+|f|^p$,
    $$\int |f|^{p'}\d\mu \leq \int 1+|f|^p \d\mu = \mu(\Omega) + \int |f|^p \d\mu < \infty.$$
    
    Now, for any $f,g\in\mathcal{L}^{p}(\mu)$, let $c=\norm{f-g}_{p}$. Then
    \begin{align*}
        |f-g|^{p'} &= |f-g|^{p'}\indic_{|f-g|\leq c} + |f-g|^{p'}\indic_{|f-g|> c} \\
        &\leq c^{p'} + c^{p'-p}|f-g|^p
    \end{align*}
    
    This implies that
    \begin{align*}
        \norm{f-g}_{p'} &\leq c\left(\int 1+c^{-p}|f-g|^p\d\mu\right)^{1/p'} \\
        &= \norm{f-g}_{p}(1+\mu(\Omega))^{1/p'}.
    \end{align*}
    This completes the proof.
\end{proof}

\subsection{Monotone Convergence and Fatou's Lemma}

Under what conditions can we exchange the limit and the integral? We answered this question in part in \cref{properties of integral 2}(b). Over the course of this subsection, we attempt to answer this.

\begin{theorem}[Monotone Convergence, Beppo-Levi Theorem]
\label{Monotone Convergence Theorem}
Let $f_1,f_2,\ldots\in\mathcal{L}^1(\mu)$ and $f:\Omega\to\overline{\mathbb{R}}$ be measurable. Assume that $f_n\uparrow f$ almost everywhere. Then
$$\lim_{n\to\infty}\int f_n\d\mu = \int f\d\mu$$
where both sides can equal $\infty$.
\end{theorem}
\begin{proof}
    Let $N$ be a set such that $\mu(N)=0$ and $f_n(\omega)\uparrow f(\omega)$ for all $\omega\in N^c$. For each $n\in\mathbb{N}$, define
    $$f_n'=(f_n-f_1)\indic_{N^c}\text{ and }f'=(f-f_1)\indic_{N^c}.$$
    Note that $f_n'\uparrow f'$ and each of these functions are non-negative. By \cref{properties of integral 2}(b), $\int f_n'\d\mu\uparrow\int f'\d\mu$. Then by \cref{properties of integral final}(a),
    \begin{align*}
        \lim_{n\to\infty} \int f_n\d\mu &= \lim_{n\to\infty}\left(\int f_n'\d\mu + \int f_1\d\mu\right) \\
        &= \int f'\d\mu + \int f_1\d\mu \\
        &= \int f\d\mu.
    \end{align*}
\end{proof}

\begin{theorem}[Fatou's Lemma]
\label{fatous lemma}
    Let $f\in\mathcal{L}^1(\mu)$ and let $f_1,f_2,\ldots$ be measurable with $f_n\geq f$ almost everywhere for all $n\in\mathbb{N}$. Then
    $$\int \liminf_{n\to\infty} f_n\d\mu \leq \liminf_{n\to \infty} \int f_n\d\mu$$
\end{theorem}
\begin{proof}
    Considering $f_n-f$ for each $n\in\mathbb{N}$, we may assume each $f_n$ to be non-negative almost everywhere. For each $m\in\mathbb{N}$, consider $g_m=\inf_{n\geq m}f_n$. Note that $g_m\uparrow \liminf_{n\to\infty} f_n$. Thus, using \cref{properties of integral final}(a) and \cref{Monotone Convergence Theorem},
    $$\int \liminf_{n\to\infty} f_n = \int \lim_{m\to\infty} g_m\d\mu = \lim_{m\to\infty}\int g_m\d\mu \leq \liminf_{n\to\infty}\int f_n\d\mu.$$
\end{proof}

In the above theorem, we require an integrable $f$ for the statement to hold. This $f$ is called a ``minorant".

% \vspace{2mm}
% An example to show this is the following, known as Peterburg's game.

% Consider a gamble in a casino where you double your bet with probability $p\leq\frac{1}{2}$ and lose it with probability $1-p$. We gamble over and over. This can be modelled by the probability space $(\Omega,\mathcal{A},\textbf{P})$ where $\Omega=\{-1,1\}^\mathbb{N}$, $\mathcal{A}=(2^{\{-1,1\}})^{\otimes\mathbb{N}}$, and $\textbf{P}=((1-p)\delta_{-1} + p\delta_1)^\mathbb{N}$. Let us denote by $D_n:\Omega\to\{-1,1\}$ the output of the $n$th game

% \vspace{1mm}
% If the player bets $b_i$ dollars on the $i$th game, then his total profit after the $n$th game is $S_n\sum_{i=1}^n b_iD_i$. Now, let the gambler assume the following strategy. He bets $1$ dollar on the first game. If he wins, he stops playing ($H_n=0$ for all $n\geq 2$). If he loses, he doubles the bet in the subsequent round.That is,
% $$
% H_n=
% \begin{cases}
%     0, &\text{if $D_i=1$ for some $i<n$} \\
%     1, &\text{otherwise.}
% \end{cases}
% $$
% The probability of no win until the $n$th game is $(1-p)^n$.

% Therefore, $\textbf{P}[S_n=1-2^n]=(1-p)^n$ and $\textbf{P}[S_n=1]=1-(1-p)^n$. The average expected gain is
% $$\int S_n\d{\textbf{P}} = (1-2^n)(1-p)^n + (1-(1-p)^n)=1-(2(1-p))^n\leq 0.$$
% Define
% $$
% S=
% \begin{cases}
%     -\infty, &\text{if $-1=D_1=D_2=\cdots$} \\
%     1, &\text{otherwise.}
% \end{cases}
% $$

% Then $\lim_{n\to\infty} S_n = S$ almost surely. However, $\lim_{n\to\infty}\int S_n\d{\textbf{P}} < \lim_{n\to\infty}\int S\d{\textbf{P}}$ since $S=1$ almost surely. 

% By Fatou's Lemma, this is only possible if there is no integrable minorant $f$ for $(S_n)_{n\in\mathbb{N}}$.

% Indeed, letting $\tilde S = \inf \{S_n:{n\in\mathbb{N}}\}$,
% $$\textbf{P}[\tilde S = 1-2^{n-1}]=\textbf{P}[D_1=D_2=\cdots=D_{n-1}=-1\text{ and }D_n=1]=p(1-p)^{n-1}.$$
% Therefore,
% $$\int \tilde S\d{\textbf{P}}=\sum_{n=1}^\infty (1-2^{n-1})p(1-p)^{n-1}=-\infty.$$

\clearpage