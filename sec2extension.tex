\section{Some Results}

\subsection{Outer Measure}

\begin{lemma}
\label{uniquely defined by base pi sys}
    Let $(\Omega,\mathcal{A},\mu)$ be a $\sigma$-finite measure space and $\mathcal{E}\subseteq\mathcal{A}$ be a $\pi$-system that generates $\mathcal{A}$. Assume there exists sequence $\Omega_1,\Omega_2\ldots\in\mathcal{E}$ such that $\bigcup_{i=1}^\infty\Omega_i=\Omega$ and $\mu(\Omega_i)<\infty$ for all $i\in\mathbb{N}$. Then $\mu$ is uniquely determined by the values $\mu(E)$, $E\in\mathcal{E}$.
    
    If $\Omega\in\mathcal{A}$ and $\mu(\Omega)=1$, then the existence of the sequence $(\Omega_n)_{n\in\mathbb{N}}$ is not required.
\end{lemma}
\begin{proof}
    Let $\nu$ be a $\sigma$-finite measure on $(\Omega,\mathcal{A})$ such that $\mu(E)=\nu(E)$ for all $E\in\mathcal{E}$.
    
    \vspace{1mm}
    Let $E\in\mathcal{E}$ with $\mu(E)<\infty$. Consider
    $$\mathcal{D}_E=\{A\in\mathcal{A}: \mu(A\cap E)=\nu(A\cap E)\}.$$
    
    We claim that $\mathcal{D}_E$ is a $\lambda$-system. We shall prove this by checking each of the conditions of \cref{defLamSystem}.
    
    \begin{enumerate}[(a)]
        \item Clearly, $\Omega\in\mathcal{D}_E$.
        \item Let $A,B\in\mathcal{D}_E$ with $B\subseteq A$. Then
        \begin{align*}
            \mu((A\setminus B)\cap E) &= \mu(A\cap E) - \mu(B\cap E)\quad\text{(using \cref{properties of content})} \\
            &= \nu(A\cap E) - \nu(B\cap E) \\
            &= \nu((A\setminus B)\cap E).
        \end{align*}
        That is, $(A\setminus B)\in\mathcal{D}_E$.
        \item Let $A_1,A_2,\ldots\in\mathcal{D}_E$ be mutually disjoint sets. Then
        \begin{align*}
            \mu\left(\left(\biguplus_{i=1}^\infty A_i\right)\cap E\right) &= \sum_{i=1}^\infty \mu(A_i\cap E) \\
            &= \sum_{i=1}^\infty\nu(A_i\cap E) \\
            &= \nu\left(\left(\biguplus_{i=1}^\infty A_i\right)\cap E\right).
        \end{align*}
        Therefore, $\biguplus_{i=1}^\infty A_i\in\mathcal{D}_E$ and $\mathcal{D}_E$ is a $\lambda$-system.
    \end{enumerate}
    As $\mathcal{E}\subseteq\mathcal{D}_E$ (Why?), $\delta(\mathcal{E})\subseteq\mathcal{D}_E$. Since $\mathcal{E}$ is a $\pi$-system, \cref{dynkins pi lam theorem} implies that
    $$\mathcal{A}\supseteq\mathcal{D}_E\supseteq\delta(\mathcal{E})=\sigma(\mathcal{E})=\mathcal{A}.$$
    Hence $\mathcal{D}_E=\mathcal{A}$.
    
    Therefore, $\mu(A\cap E)=\nu(A\cap E)$ for any $A\in\mathcal{A}$ and $E\in\mathcal{E}$ with $\mu(E)<\infty$.
    
    Now, let $\Omega_1,\Omega_2,\ldots\in\mathcal{E}$ be a sequence such that $\bigcup_{i=1}^\infty\Omega_i=\Omega$ and $\mu(\Omega_i)<\infty$ for all $i\in\mathbb{N}$. Let $E_0=\emptyset$ and $E_n=\bigcup_{i=1}^n\Omega_i$ for each $n\in\mathbb{N}$. Note that
    $$E_n=\biguplus_{i=1}^n(E_{i-1}^c\cap \Omega_i).$$
    Therefore for any $A\in\mathcal{A}$ and $n\in\mathbb{N}$,
    \begin{align*}
        \mu(A\cap E_n) &= \sum_{i=1}^n\mu((A\cap E_{i-1}^c)\cap\Omega_i) \\
        &= \sum_{i=1}^n\nu((A\cap E_{i-1}^c)\cap\Omega_i) = \nu(A\cap E_n).
    \end{align*}
    
    Now, since $E_n\uparrow\Omega$ and $\mu,\nu$ are lower semicontinuous (by \cref{tripledoubleEquivalence}),
    \begin{align*}
        \mu(A) &= \lim_{n\to\infty}\mu(A\cap E_n) \\
        &= \lim_{n\to\infty}\nu(A\cap E_n) = \nu(A)
    \end{align*}
    
    This proves the result.
    
    \vspace{2mm}
    The second part of the theorem is trivial as $\mathcal{E}\cup\{\Omega\}$ is a $\pi$-system that generates $\mathcal{A}$. Hence one can choose the constant sequence $E_n=\Omega, n\in\mathbb{N}$.
\end{proof}

\begin{definition}[Outer Measure]
    A function $\mu^*:2^\Omega\to[0,\infty]$ is called an \textit{outer measure} if
    \begin{enumerate}[(i)]
        \item $\mu^*(\emptyset)=0$,
        \item $\mu^*$ is monotone, and
        \item $\mu^*$ is $\sigma$-subadditive.
    \end{enumerate}
\end{definition}

\begin{lemma}
\label{set of countable coverings outer measure}
    Let $\mathcal{A}\subseteq 2^\Omega$ be an arbitrary class of sets with $\emptyset\in\mathcal{A}$ and let $\mu$ be a nonnegative set function on $\mathcal{A}$ with $\mu(\emptyset)=0$. For $A\subseteq\Omega$, define the set of countable coverings $\mathcal{F}$ with sets $F\in\mathcal{A}$
    $$\mathcal{U}(A)=\left\{\mathcal{F}\subseteq\mathcal{A} : \mathcal{F}\text{ is countable and }A\subseteq\bigcup_{F\in\mathcal{F}}F\right\}.$$
    Define
    $$\mu^*(A)=\inf\left\{\sum_{F\in\mathcal{F}}\mu(F) : \mathcal{F}\in\mathcal{U}(A) \right\}$$
    where $\inf\emptyset=\infty$. Then $\mu^*$ is an outer measure. If $\mu$ is $\sigma$-subadditive then $\mu^*(A)=\mu(A)$ for all $A\in\mathcal{A}$.
\end{lemma}
\begin{proof}
    Let us check each of the three conditions in the definition of an outer measure.
    \begin{enumerate}[(a)]
        \item Since $\emptyset\in\mathcal{A}$, we have $\{\emptyset\}\in\mathcal{U}(\emptyset)$ and hence $\mu(\emptyset)=0$.
        
        \item If $A\subseteq B$, then $\mathcal{U}(A)\subseteq\mathcal{U}(B)$, and hence $\mu^*(A)\leq\mu^*(B)$.
        
        \item Let $A,A_1,A_2,\ldots\subseteq\Omega$ such that $A\subseteq\bigcup_{i=1}^\infty A_i$. We claim that $\mu^*(A)\leq\sum_{i=1}^\infty \mu^*(A_i)$.
        
        Without loss of generality, assume that $\mu^*(A_i)<\infty$ and hence $\mathcal{U}(A_i)\neq\emptyset$ for all $i\in\mathbb{N}$. Fix some $\varepsilon>0$. Now, for every $n\in\mathbb{N}$, we may choose a covering $\mathcal{F}_n\in\mathcal{U}(A_n)$ such that
        $$\sum_{F\in\mathcal{F}_n}\mu(F)\leq\mu^*(A_n)+\varepsilon2^{-n}.$$
        Then let $\mathcal{F}=\bigcup_{n=1}^\infty \mathcal{F}_n\in\mathcal{U}(A)$.
        $$\mu^*(A)\leq\sum_{F\in\mathcal{F}}\mu(F)\leq\sum_{n=1}^\infty\sum_{F\in\mathcal{F}_n}\mu(F)\leq\sum_{n=1}^\infty \mu^*(A_n) + \varepsilon.$$
        
        This proves the first part of the result.
    \end{enumerate}
    
    To prove the next part of the result, first note that since $\{A\}\in\mathcal{U}(A)$, we have $\mu^*(A)\leq\mu(A)$. If $\mu$ is $\sigma$-subadditive, then for any $\mathcal{F}\in\mathcal{U}(A)$,
    $$\sum_{F\in\mathcal{F}}\mu(F)\geq\mu(A).$$
    It follows that $\mu^*(A)\geq\mu(A).$
\end{proof}

\begin{definition}[$\mu^*$-measurable sets]
    Let $\mu^*$ be an outer measure. A set $A\in 2^\Omega$ is called \textit{$\mu^*$-measurable} if
    $$\mu^*(A\cap E) + \mu^*(A^c\cap E) = \mu^*(E)\text{ for any }E\in 2^\Omega.$$
    We write $\mathcal{M}(\mu^*)=\{A\subseteq\Omega:A\text{ is }\mu^*\text{-measurable}\}.$
\end{definition}

\begin{lemma}
\label{mu measurable iff leq}
    $A\in\mathcal{M}(\mu^*)$ if and only if
    $$\mu^*(A\cap E) + \mu^*(A^c\cap E) \leq \mu^*(E)\text{ for any }E\in 2^\Omega.$$
\end{lemma}
\begin{proof}
    As $\mu^*$ is subadditive, we trivially have
    $$\mu^*(A\cap E) + \mu^*(A^c\cap E) \geq \mu^*(E)\text{ for any }E\in 2^\Omega.$$
    The result follows.
\end{proof}

\begin{lemma}
    $\mathcal{M}(\mu^*)$ is an algebra.
\end{lemma}
\begin{proof}
    We shall check the conditions given in the definition of an algebra \cref{defAlgebra}.
    \begin{enumerate}[(a)]
        \item We clearly have $\Omega\in\mathcal{M}(\mu^*)$.
        \item By definition, $\mathcal{M}(\mu^*)$ is closed under complements.
        \item We must check that $\mathcal{M}(\mu^*)$ is closed under intersections. Let $A,B\in\mathcal{M}(\mu^*)$ and $E\subseteq\Omega$. Then
        \begin{align*}
            \mu^*((A\cap B)\cap E)+\mu^*((A\cap B)^c\cap E) &= \mu^*((A\cap B)\cap E) \\ &\hspace{4mm}+\mu^*((A\cap B^c\cap E)\cup(A^c\cap B\cap E)\cup(A^c\cap B^c\cap E)) \\
            &\leq \mu^*(A\cap (B\cap E))+\mu^*(A\cap (B^c\cap E))\\ &\hspace{4mm}+\mu^*(A^c\cap (B\cap E)) + \mu^*(A^c\cap (B^c\cap E)) \\
            &= \mu^*(B\cap E) + \mu^*(B^c\cap E)\quad\text{(since $A\in\mathcal{M}(\mu^*)$)} \\
            &= \mu^*(E). \quad\text{(since $B\in\mathcal{M}(\mu^*)$)}
        \end{align*}
    \end{enumerate}
    This proves the result.
\end{proof}

\begin{lemma}
\label{outer measure is sig additive on M}
    An outer measure $\mu^*$ is $\sigma$-additive on $\mathcal{M}(\mu^*)$.
\end{lemma}
\begin{proof}
    Let $A,B\in\mathcal{M}(\mu^*)$ with $A\cap B=\emptyset$. Then
    \begin{align*}
        \mu^*(A\cup B) &= \mu^*(A\cap(A\cup B)) + \mu^*(A^c\cap(A\cup B)) \\
        &= \mu^*(A) + \mu^*(B).
    \end{align*}
    That is, $\mu^*$ is additive (and thus a content) on $\mathcal{M}(\mu^*)$. Since $\mu^*$ is $\sigma$-subadditive, \cref{tripledoubleEquivalence} gives the required result.
\end{proof}

\begin{lemma}
    If $\mu^*$ is an outer measure, $\mathcal{M}(\mu^*)$ is a $\sigma$-algebra.
\end{lemma}
\begin{proof}
    We have already shown that $\mathcal{M}(\mu^*)$ is an algebra (and thus a $\pi$-system). Using \cref{cap closed lam sys}, it is sufficient to show that $\mathcal{M}(\mu^*)$ is a $\lambda$-system.
    
    Let $A_1,A_2,\ldots\in\mathcal{M}(\mu^*)$ be mutually disjoint sets and let $A=\biguplus_{i=1}^\infty A_i$. Further, for each $n\in\mathbb{N}$, let $B_n=\biguplus_{i=1}^n A_i$. We must show that $M\in\mathcal{M}(\mu^*)$.
    
    For any $E$ and valid $n\in\mathbb{N}$, we have
    \begin{align*}
        \mu^*(E\cap B_{n+1}) &= \mu^*((E\cap B_{n+1})\cap B_n) + \mu^*((E\cap B_{n+1})\cap B_n^c) \\
        &= \mu^*(E\cap B_n) + \mu^*(E\cap A_{n+1}).
    \end{align*}
    By a simple induction, it follows that
    $$\mu(E\cap B_{n})=\sum_{i=1}^n \mu^*(E\cap A_i).$$
    Since $\mu^*$ is monotonic, we have
    \begin{align*}
        \mu^*(E) &= \mu^*(E\cap B_n) + \mu^*(E\cap B_n^c) \\
        &\geq \mu^*(E\cap B_n) + \mu^*(E\cap A^c) \\
        &= \sum_{i=1}^n\mu^*(E\cap A_i) + \mu^*(E\cap A^c).
    \end{align*}
    Letting $n\to\infty$ and using the fact that $\mu^*$ is $\sigma$-subadditive, we have
    \begin{align*}
        \mu^*(E) &\geq \sum_{i=1}^\infty\mu^*(E\cap A_i) + \mu^*(E\cap A^c) \\
        &\geq \mu^*(E\cap A) + \mu^*(E\cap A^c)
    \end{align*}
    Therefore, $A\in\mathcal{M}(\mu^*)$ and this completes the proof.
\end{proof}

\subsection{The Approximation and Extension Theorems}

\begin{theorem}[Approximation Theorem for Measures]
\label{Approximation Thm for Measures}
    Let $\mathcal{A}\subseteq 2^\Omega$ be a semiring and let $\mu$ be a measure on $\sigma(\mathcal{A})$ that is $\sigma$-finite on $\mathcal{A}$.
    For any $A\in\sigma(\mathcal{A})$ with $\mu(\mathcal{A})<\infty$ and any $\varepsilon>0$, there exists $n\in\mathbb{N}$ and mutually disjoint sets $A_1,A_2,\ldots,A_n\in \mathcal{A}$ such that $\mu\left(A\triangle\bigcup_{i=1}^n A_n\right)<\varepsilon$.
\end{theorem}
\begin{proof}
    Consider the outer measure $\mu^*$ as defined in \cref{set of countable coverings outer measure}. Note that by \cref{set of countable coverings outer measure} and \cref{uniquely defined by base pi sys}, $\mu$ and $\mu^*$ are equal on $\sigma(\mathcal{A})$. By the definition of $\mu^*$, for any $A\in\mathcal{A}$, there exists a covering $B_1,B_2,\ldots\in\mathcal{A}$ of $A$ such that
    $$\mu(A)\geq\sum_{i=1}^\infty\mu(B_i) - \varepsilon/2.$$
    Since $\mu(A)<\infty$, there exists some $n\in\mathbb{N}$ such that $\sum_{i=n+1}^\infty \mu(B_i) < \varepsilon/2$. Now, let $D=\bigcup_{i=1}^n B_i$ and $E=\bigcup_{i=n+1}^\infty B_i$. We have
    \begin{align*}
        A\triangle D &= (D\setminus A)\cup(A\setminus D) \\
        &\subseteq (D\setminus A)\cup (A\setminus (D\cup E))\cup E \\
        &\subseteq (A\triangle (D\cup E))\cup E.
    \end{align*}
    This together with the fact that $A\subseteq\bigcup_{i=1}^\infty B_i$ implies that
    \begin{align*}
        \mu(A\triangle D) &\leq \mu(A\triangle (D\cup E)) + \mu(E) \\
        &\leq  \mu\left(\bigcup_{i=1}^\infty B_i\right) - \mu(A) + \frac{\varepsilon}{2} \\
        &\leq \varepsilon.
    \end{align*}
    Now define $A_1=B_1$ and for each $i\geq 2$, $A_i = B_i\setminus \bigcup_{j=1}^{i=1} B_j$. By definition, $A_1,A_2\ldots$ are mutually disjoint. This proves the result.
    
\end{proof}

The following theorem allows us to ``extend" measures from a semiring to the $\sigma$-algebra generated by it. This allows us to define measures over an entire $\sigma$-algebra by defining its values over just a semiring that generates it.

\begin{theorem}[Measure Extension Theorem]
\label{MeasureExtensionTh}
    Let $\mathcal{A}$ be a semiring and let $\mu:\mathcal{A}\to[0,\infty]$ be an additive, $\sigma$-subadditive and $\sigma$-finite set function with $\mu(\emptyset)=0$. Then there is a unique $\sigma$-finite measure $\tilde\mu:\sigma(\mathcal{A})\to[0,\infty]$ such that $\tilde\mu(A)=\mu(A)$ for all $A\in\mathcal{A}$.
\end{theorem}
\begin{proof}
    Since $\mathcal{A}$ is a $\pi$-system, if such a $\tilde\mu$ exists, it is uniquely defined due to \cref{uniquely defined by base pi sys}.
    
    We shall explicitly construct a function that satisfies the given conditions. In order to do so, define as in \cref{set of countable coverings outer measure}
    $$\mu^*(A)=\inf\left\{\sum_{F\in\mathcal{F}}\mu(F) : \mathcal{F}\in\mathcal{U}(A) \right\}\text{ for any $A\subseteq\Omega$.}$$
    
    By \cref{set of countable coverings outer measure}, $\mu^*$ is an outer measure and $\mu^*(A)=\mu(A)$ for any $A\in\mathcal{A}$.
    
    \vspace{2mm}
    We first claim that $\mathcal{A}\subseteq\mathcal{M}(\mu^*)$.
    
    To prove this, let $A\in\mathcal{A}$ and $E\subseteq\Omega$ with $\mu^*(E)<\infty$. Fix some $\varepsilon>0$. Then by the definition of $\mu^*$, there exists a sequence $E_1,E_2,\ldots\in\mathcal{A}$ such that
    $$E\subseteq\bigcup_{i=1}^\infty E_i\text{ and }\sum_{i=1}^\infty \mu(E_i)\leq \mu^*(E)+\varepsilon.$$
    
    For each $n$, define $B_n=E_n\cap A$. Since $\mathcal{A}$ is a semiring, there exists for each $n$ some $m_n\in\mathbb{N}$ and mutually disjoint sets $C_{n, 1},C_{n, 2},\ldots,C_{n, m_n}$ such that
    $$E_n\setminus A = E_n\setminus B_n = \biguplus_{i=1}^{m_n}C_{n, i}$$
    Then we have that
    \begin{align*}
        E\cap A &\subseteq \bigcup_{n=1}^\infty B_n, \\
        E\cap A^c &\subseteq \bigcup_{n=1}^\infty\biguplus_{i=1}^{m_n}C_{n,i},\text{ and } \\
        E_n &= B_n\uplus\biguplus_{i=1}^{m_n}C_{n,i}.
    \end{align*}
    
    This implies that
    \begin{align*}
        \mu^*(E\cap A) + \mu^*(E\cap A^c) &\leq \sum_{n=1}^\infty\mu(B_n) + \sum_{n=1}^\infty\sum_{i=1}^{m_n}\mu(C_{n,i}) \quad\text{(since $\mu$ is $\sigma$-subadditive)} \\
        &= \sum_{n=1}^\infty\left(\mu(B_n) + \sum_{i=1}^{m_n}\mu(C_{n,i})\right) \\
        &= \sum_{n=1}^\infty \mu(E_n) \quad\text{(since $\mu$ is additive)} \\
        &   \leq \mu^*(E) + \varepsilon.
    \end{align*}
    
    \cref{mu measurable iff leq} implies that $A\in\mathcal{M}(\mu^*)$, that is, $\mathcal{A}\subseteq\mathcal{M}(\mu^*)$. This in turn in implies that $\sigma(\mathcal{A})\subseteq\mathcal{M}(\mu^*)$. Define the required function by $\tilde\mu:\sigma(\mathcal{A})\to[0,\infty]$, $A\mapsto\mu^*(A)$. By \cref{outer measure is sig additive on M}, $\tilde\mu$ is $\sigma$-additive. Since $\mu$ is $\sigma$-finite, $\tilde\mu$ is $\sigma$-finite as well. This proves the result.
\end{proof}

\subsection{Important Examples of Measures}

Now that we have the Measure Extension Theorem, we may introduce the Lebesgue-Stieltjes measure, a very useful measure on $(\mathbb{R},\mathcal{B}(\mathbb{R}))$, which is given as follows.

\begin{definition}[Lebesgue-Stieltjes Measure]
\label{defLebStielMeasure}
    Let $F:\mathbb{R}\to\mathbb{R}$ be monotone increasing and right continuous. The measure $\mu_F$ on $(\mathbb{R},\mathcal{B}(\mathbb{R}))$ defined by 
    $$\mu_F((a,b])= F(b)-F(a)\text{ for all $a,b\in\mathbb{R}$ such that $a<b$}$$
    is called the \textit{Lebesgue-Stieltjes measure} with distribution function $F$.
\end{definition}

The Lebesgue-Stieltjes measure is well-defined due to the Measure Extension Theorem \cref{MeasureExtensionTh}.

\vspace{1mm}
To see this more clearly, let $\mathcal{A}=\{(a,b\,]:a,b\in\mathbb{R}\text{ and }a\leq b\}$. It may be checked that $\mathcal{A}$ is a semiring. Further, $\sigma(\mathcal{A})=\mathcal{B}(\mathbb{R})$. Now, define the function $\tilde\mu_F:\mathcal{A}\to[0,\infty)$ by $(a,b]\mapsto F(b)-F(a)$. Clearly $\tilde\mu_F(\emptyset)=0$ and the function is additive. It remains to check that $\tilde\mu_F$ is $\sigma$-subadditive.

\vspace{1mm}
Let $(a,b],(a_1,b_1],(a_2,b_2],\ldots\in\mathcal{A}$ such that $(a,b\,]\subseteq\bigcup_{i=1}^\infty (a_i, b_i]$. Fix some $\varepsilon>0$ and choose $a_\varepsilon\in(a,b)$ such that
$$F(a_\varepsilon)-F(a)< \varepsilon/2\implies \tilde\mu_F((a,b]) - \tilde\mu_F((a_\varepsilon,b\,]) < \varepsilon/2.$$
It is possible to choose such an $\varepsilon$ due to the right continuity of $F$. Also, for any $k\in\mathbb{N}$, choose $b_{k, \varepsilon}$ such that $$F(b_{k,\varepsilon})-F(b_k)<\varepsilon 2^{-k-1}\implies \tilde\mu_F((a_k,b_{k,\varepsilon}]) - \tilde\mu_F((a_k,b_k]) < \varepsilon 2^{-k-1}.$$
We now have
$$[a_\varepsilon, b\,]\subseteq (a,b]\subseteq \bigcup_{i=1}^\infty (a_k,b_k]\subseteq \bigcup_{k=1}^\infty (a_k,b_{k,\varepsilon}]$$
Due to the compactness of $[a_\varepsilon,b]$, there then exists some $k_0\in\mathbb{N}$ such that
$$(a_{\varepsilon},b\,]\subseteq\bigcup_{k=1}^{k_0}(a_k,b_{k,\varepsilon}].$$
This implies that
\begin{align*}
    \tilde\mu_F((a,b\,]) &\leq \frac{\varepsilon}{2} + \tilde\mu_F((a,b]) \\
    &\leq \frac{\varepsilon}{2} + \sum_{k=1}^{k_0} \tilde\mu_F((a_k,b_{k,\varepsilon}]) \\
    &\leq \frac{\varepsilon}{2} + \sum_{k=1}^{k_0} \left(\tilde\mu_F((a_k,b_k]) + \varepsilon2^{-k-1}\right) \\
    &\leq \varepsilon + \sum_{k=1}^{\infty} \tilde\mu_F((a_k,b_k])
\end{align*}
As this is true for any choice of $\varepsilon$, $\tilde\mu_F$ is $\sigma$-subadditive.

\vspace{2mm}
Then the extension of $\tilde\mu_F$ uniquely to a $\sigma$-finite measure is guaranteed by \cref{MeasureExtensionTh}. This measure is known as the Lebesgue-Stieltjes measure.

\vspace{2mm}
The measure that results when the function $F$ is equal to the identity function is referred to the \textit{Lebesgue measure} on $\mathbb{R}^1$. Similar to this, we can define the Lebesgue measure in general as follows.

\begin{definition}[Lebesgue Measure]
    There exists a unique measure $\lambda^n$ on $(\mathbb{R}^n,\mathcal{B}(\mathbb{R}^n))$ such that for all $a,b\in\mathbb{R}^n$ with $a<b$,
    $$\lambda^n((a,b])=\prod_{i=1}^n (b_i-a_i).$$
    $\lambda^n$ is called the \textit{Lebesgue measure} on $(\mathbb{R}^n, \mathcal{B}(\mathbb{R}^n))$ or the \textit{Lebesgue-Borel measure}.
\end{definition}

\vspace{2mm}
Let $E$ be a finite nonempty set and $\Omega = E^\mathbb{N}$. If $\omega_1,\omega_2,\ldots,\omega_n\in E$, we define the following.
$$[\omega_1,\omega_2,\ldots,\omega_n]=\{\omega'\in\Omega : \omega'_i=\omega_i\text{ for }i\in[n]\}.$$
This represents the set of all sequences whose first $n$ elements are $\omega_1,\omega_2,\ldots,\omega_n$.

\begin{theorem}[Finite Products of Measures]
        Let $n\in\mathbb{N}$ and $\mu_1,\mu_2,\ldots,\mu_n$ be Lebesgue-Stieltjes measures on $(\mathbb{R},\mathcal{B}(\mathbb{R}))$. Then there exists a unique $\sigma$-finite measure $\mu$ on $(\mathbb{R}^n, \mathcal{B}(\mathbb{R}^n))$ such that for all $a,b\in\mathbb{R}^n$ with $a<b$,
        $$\mu((a,b])=\prod_{i=1}^n \mu_i((a_i,b_i])$$
        We call $\mu$ the \textit{product measure of $\mu_1,\mu_2,\ldots,\mu_n$} and denote it by $\bigotimes_{i=1}^n\mu_i$.
\end{theorem}

The proof of the above is similar to that of \cref{MeasureExtensionTh}. We choose intervals $(a,b_\varepsilon]$ and so on such that $\mu((a,b_\varepsilon])<\mu((a,b])+\varepsilon$. Such $b_\varepsilon$ exists due to the right continuity of each of the $F_i$s corresponding to each of the $\mu_i$s.

\clearpage