\section{Notation}

$\mathbb{N}$ represents the set $\{1,2,\ldots\}$.

\vspace{1mm}
$\mathbb{N}_0$ represents the set $\{0,1,2,\ldots\}$.

\vspace{1mm}
For $x\in\mathbb{R}$ and $n\in\mathbb{N}$,
$$\binom{x}{r}=\frac{x(x-1)\cdots(x-r+1)}{r!}$$
is the generalised binomial coefficient.

\vspace{2mm}
For $n\in\mathbb{N}$, we denote $\{1,2,\ldots,n\}$ as $[n]$ and $\{0,1,2,\ldots,n\}$ as $[n]_0$

\vspace{2mm}
For $a\in\mathbb{R}^n$, we denote the $i$th coordinate of $a$ by $a_i$ for each $i=1,2,\ldots,n$.

\vspace{1mm}
For $a,b\in\mathbb{R}^n$, we write $a<b$ if $a_i<b_i$ for each $i=1,2,\ldots,n$.

\vspace{2mm}
Let $(a_n)_{n\in\mathbb{N}}$ be a sequence of reals. Then
\begin{align*}
    \limsup_{n\to\infty}a_n &= \lim_{n\to\infty}\left(\sup_{m\geq n}a_n\right) =    \inf_{n\geq 0}\left(\sup_{m\geq n}a_n\right) \\
    \liminf_{n\to\infty}a_n &= \lim_{n\to\infty}\left(\inf_{m\geq n}a_n\right) =    \sup_{n\geq 0}\left(\inf_{m\geq n}a_n\right)
\end{align*}

\vspace{2mm}
Let $A$ and $B$ be two sets. We denote by
$$A\triangle B=(A\setminus B)\cup(B\setminus A)$$
the \textit{symmetric difference} of $A$ and $B$.

\vspace{2mm}
If sets $A$ and $B$ are disjoint, we represent their union as $A\uplus B$ (similar to the $\oplus$ notation in linear algebra).   

\vspace{2mm}
If $f,g:\Omega\to\overline{\mathbb{R}}$ such that $f(\omega)\leq g(\omega)$ for all $\omega\in\Omega$, we write $f\leq g$. We similarly write $f<g$, $f\leq 0$ and other expressions.

\vspace{2mm}
If $\Omega$ is a set and $f:\Omega\to\overline{\mathbb{R}}$, then we define the functions $f^+,f^-:\Omega\to\overline{\mathbb{R}}$ by
$$f^+=\max\{0,f\}\text{ and }f^-=\max\{0,-f\}.$$
\clearpage