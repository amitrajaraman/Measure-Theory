\section{Measure Theory}

Before beginning a rigorous study of probability theory, it is necessary to understand some parts of basic measure theory.

\subsection{Classes of Sets}

\vspace{2mm}
Let $\Omega$ be a non-empty set and $\mathcal{A}\subseteq2^\Omega$, where $2^\Omega$ is the power set of $\Omega$. Then

\begin{definition}
    $\mathcal{A}$ is called
    \begin{itemize}
        \item $\cap$-closed (closed under intersections) or a $\pi$-system if $A\cap B\in \mathcal{A}$ for all $A,B\in\mathcal{A}$.
        \item $\sigma$-$\cap$-closed (closed under countable intersections) if $\bigcap_{i=1}^\infty A_i\in\mathcal{A}$ for any choice of countably many sets\\ $A_1, A_2, \ldots\in\mathcal{A}$.
        \item $\cup$-closed (closed under unions) if $A\cup B\in \mathcal{A}$ for all $A,B\in\mathcal{A}$.
        \item $\sigma$-$\cup$-closed (closed under countable unions) if $\bigcup_{i=1}^\infty A_i\in\mathcal{A}$ for any choice of countably many sets $A_1, A_2, \ldots\in\mathcal{A}$.
        \item $\setminus$-closed (closed under differences) if $A\setminus B\in\mathcal{A}$ for all $A,B\in\mathcal{A}$.
        \item closed under complements if $A^c=\Omega\setminus A\in\mathcal{A}$ for all $A\in\mathcal{A}$.
    \end{itemize}
\end{definition}

\begin{theorem}
\label{cupclosediffcapclosed}
    Let $\mathcal{A}$ be closed under complements. Then $\mathcal{A}$ is $\cup$-closed ($\sigma$-$\cup$-closed) if and only if $\mathcal{A}$ is closed $\cap$-closed ($\sigma$-$\cap$-closed).
\end{theorem}

The above is relatively straightforward to prove using De Morgan's Laws.

\begin{theorem}
\label{if setminus closed}
    Let $\mathcal{A}$ be $\setminus$-closed. Then
    \begin{enumerate}[(a)]
        \item $\mathcal{A}$ is $\cap$-closed,
        \item if $\mathcal{A}$ is $\sigma$-$\cup$-closed, then $\mathcal{A}$ is $\sigma$-$\cap$-closed.
        \item Any countable union of sets in $\mathcal{A}$ can be expressed as a countable union of pairwise disjoint sets in $\mathcal{A}$.
    \end{enumerate}
\end{theorem}
\begin{proof}
~
\begin{enumerate}[(a)]
    \item For $A,B\in\mathcal{A}$, $A\cap B=A\setminus(A\setminus B)\in\mathcal{A}$.
    \item Let $A_1,A_2,\ldots\in\mathcal{A}$. Then
    \begin{align*}
        \bigcap_{i=1}^\infty A_i &= \bigcap_{i=1}^\infty (A_1\cap A_i) \\
        &= \bigcap_{i=1}^\infty A_1\setminus (A_1\setminus A_i) \\
        &= A_1\setminus\bigcup_{i=1}^\infty (A_1\setminus A_i).
    \end{align*}
    \item Let $A_1,A_2,\ldots\in\mathcal{A}$. We then have
    $$\bigcup_{i=1}^\infty A_i = A_1 \uplus (A_2\setminus A_1)\uplus ((A_3\setminus A_2)\setminus A_1)\uplus\cdots$$
    The result follows.
    \end{enumerate}
\end{proof}

This equivalence between $\cap$ and $\cup$ if the class is $\setminus$-closed is apparent from De Morgan's laws.

\begin{definition}[Algebra]
\label{defAlgebra}
    A class of sets $\mathcal{A}\subseteq2^\Omega$ is called an \textit{algebra} if
    \begin{enumerate}[(i)]
        \item $\Omega\in\mathcal{A}$,
        \item $\mathcal{A}$ is $\setminus$-closed, and
        \item $\mathcal{A}$ is $\cup$-closed.
    \end{enumerate}
\end{definition}

\begin{definition}[$\sigma$-algebra]
\label{defSigAlgebra}
    A class of sets $\mathcal{A}\subseteq 2^\Omega$ is called a \textit{$\sigma$-algebra} if
    \begin{enumerate}[(i)]
        \item $\Omega\in\mathcal{A}$,
        \item $\mathcal{A}$ is closed under complements, and
        \item $\mathcal{A}$ is $\sigma$-$\cup$-closed.
    \end{enumerate}
\end{definition}

$\sigma$-algebras are also known as \textit{$\sigma$-fields}.

Note that any $\sigma$-algebra is an algebra (but the converse is not true).

\begin{theorem}
\label{algebra iff conditions}
    A class of sets $\mathcal{A}\subseteq2^\Omega$ is an algebra if and only if
    \begin{enumerate}[(a)]
        \item $\Omega\in\mathcal{A}$,
        \item $\mathcal{A}$ is closed under complements, and
        \item $\mathcal{A}$ is $\cap$-closed.
    \end{enumerate}
\end{theorem}

The proof of the above is left as an exercise to the reader.

\begin{definition}[Ring]
\label{defRing}
    A class of sets $\mathcal{A}\subseteq2^\Omega$ is called a \textit{ring} if
    \begin{enumerate}[(i)]
        \item $\emptyset\in\mathcal{A}$,
        \item $\mathcal{A}$ is $\setminus$-closed, and
        \item $\mathcal{A}$ is $\cup$-closed.
    \end{enumerate}
\end{definition}

Further, a ring is a \textit{$\sigma$-ring} if it is $\sigma$-$\cup$-closed.

\begin{definition}[Semiring]
\label{defSemiring}
    A class of sets $\mathcal{A}\subseteq2^\Omega$ is called a \textit{semiring} if
    \begin{enumerate}[(i)]
        \item $\emptyset\in\mathcal{A}$, 
        \item for any $A,B\in\mathcal{A}$, $A\setminus B$ is a finite union of mutually disjoint sets in $\mathcal{A}$, and
        \item $\mathcal{A}$ is $\cap$-closed.
    \end{enumerate}
\end{definition}

\begin{definition}[$\lambda$-system]
\label{defLamSystem}
    A class of sets $\mathcal{A}\subseteq2^\Omega$ is called a \textit{$\lambda$-system} (or \textit{Dynkin's $\lambda$-system}) if
    \begin{enumerate}[(i)]
        \item $\Omega\in\mathcal{A}$,
        \item for any $A,B\in\mathcal{A}$ with $B\subseteq A$, $A\setminus B\in\mathcal{A}$, and
        \item $\displaystyle\biguplus_{i=1}^\infty A_i\in\mathcal{A}$ for any choice of countably many pairwise disjoint sets $A_1,A_2,\ldots\in\mathcal{A}$.
    \end{enumerate}
\end{definition}

Among the above classes of sets, $\sigma$-algebras in particular are extremely important as we shall use them when defining probability.

\begin{theorem}
~
    \begin{enumerate}[(a)]
        \item Every $\sigma$-algebra is also a $\lambda$-system, an algebra and a $\sigma$-ring.
        \item Every $\sigma$-ring is a ring, and every ring is a semiring.
        \item Every algebra is a ring. An algebra on a finite set $\Omega$ is a $\sigma$-algebra.
    \end{enumerate}
\end{theorem}
\begin{proof}
~
    \begin{enumerate}[(a)]
        \item Let $\mathcal{A}$ be a $\sigma$-algebra. Then for any $A,B\in\mathcal{A}$, $A\setminus B = (A^c\cup B)^c \in \mathcal{A}$ and $A\cap B=(A^c\cup B^c)^c\in\mathcal{A}$, that is, $\mathcal{A}$ is $\setminus$-closed and $\cup$-closed. The result follows.
        
        \item Let $\mathcal{A}$ be a ring. Then \cref{cupclosediffcapclosed} implies that $\mathcal{A}$ is $\cap$-closed. The result follows.
        
        \item Let $\mathcal{A}$ be an algebra. With proof similar to the first part of this theorem, it is seen that $\mathcal{A}$ is $\setminus$-closed. We have $\emptyset=\Omega\setminus\Omega\in\mathcal{A}$ and thus, it is a ring. If $\Omega$ is finite, then $\mathcal{A}$ is finite. Thus any countable union of sets is a finite union of sets and the result follows.
        \end{enumerate}
\end{proof}

\begin{definition}
\label{defLimes}
    Let $A_1,A_2,\ldots$ be subsets of $\Omega$. Then
    $$\liminf_{n\to\infty}A_n:=\bigcup_{i=1}^\infty\bigcap_{j=i}^\infty A_j\text{ and }\limsup_{n\to\infty}A_n:=\bigcap_{i=1}^\infty\bigcup_{j=i}^\infty A_j$$
    are respectively called the \textit{limit inferior} and \textit{limit superior}, of the sequence $(A_n)_{n\in\mathbb{N}}$.
\end{definition}

The above may be rewritten as
\begin{align*}
    A_*:=\liminf_{n\to\infty}A_n &= \{\omega\in\Omega : |n\in\mathbb{N}:\omega\not\in A_n|<\infty\} \\
    A^*:=\limsup_{n\to\infty}A_n &= \{\omega\in\Omega : |n\in\mathbb{N}:\omega\in A_n|=\infty\}
\end{align*}

That is, $A_*$ represents the set of elements that do not appear in a finite number of sets and $A^*$ represents the set of elements that appear in an infinite number of sets. This implies that $A_*\subseteq A^*$. (Why is the opposite not necessarily true?)

\begin{definition}[Indicator function]
    Let $A$ be a subset of $\Omega$. The \textit{indicator function on $A$} is defined by
    $$\indic_A(x)
    =
    \begin{cases}
    1, & x\in A \\
    0, & x\not\in A
    \end{cases}
    $$
\end{definition}

With the above notation, it may be shown that
$$\indic_{A_*}=\liminf_{n\to\infty}\indic_{A_n}\text{ and }\indic_{A^*}=\limsup_{n\to\infty}\indic_{A_n}.$$

If $\mathcal{A}\subseteq2^\Omega$ is a $\sigma$-algebra and if $A_n\in\mathcal{A}$ for every $n\in\mathbb{N}$, then $A_*\in\mathcal{A}$ and $A^*\in\mathcal{A}$. This is clear from the fact that $\sigma$-algebras are closed under countable unions and intersections.

Proving the above statements is left as an exercise to the reader.

\begin{theorem}
\label{capofSigmaAisSigmaA}
    Let $I$ be some index set and $\mathcal{A}_i$ be a $\sigma$-algebra for each $i\in I$. Then the intersection $\mathcal{A}_I=\bigcap_{i\in I}\mathcal{A}_i$ is also a $\sigma$-algebra.
\end{theorem}
\begin{proof}
    We can prove this by using the three conditions in the definition of a $\sigma$-algebra.
    \begin{enumerate}[(i)]
        \item Since $\Omega\in \mathcal{A}_i$ for every $i\in I$, $\Omega\in \mathcal{A}_I$.
        \item Let $A\in \mathcal{A}_I$. Then $A\in \mathcal{A}_i$ for each $i\in I$ and thus $A^c\in \mathcal{A}_i$ for each $i\in I$. Therefore, $A^c\in \mathcal{A}_I$.
        \item Let $A_1,A_2,\ldots\in \mathcal{A}_I$. Then $A_n\in\mathcal{A}_i$ for each $n\in\mathbb{N}$ and $i\in I$. Thus $A=\bigcup_{n=1}^\infty A_n\in\mathcal{A}_i$ for each $i$ as well. The result follows.
    \end{enumerate}
\end{proof}

A similar statement holds for $\lambda$-systems.

\begin{theorem}
    Let $\mathcal{E}\subseteq2^\Omega$. Then there exists a smallest $\sigma$-algebra $\sigma(\mathcal{E})$ with $\mathcal{E}\subseteq\sigma(\mathcal{E})$:
    $$\sigma(\mathcal{E})=\bigcap_{\substack{\mathcal{A}\subseteq2^\Omega\text{ is a $\sigma$-algebra} \\ \mathcal{E}\subseteq\mathcal{A}}}\mathcal{A}.$$
    $\sigma(\mathcal{E})$ is called the \textit{$\sigma$-algebra generated by $\mathcal{E}$} and $\mathcal{E}$ is called a \textit{generator of $\sigma(\mathcal{E})$}. 
\end{theorem}
\begin{proof}
    $2^\Omega$ is a $\sigma$-algebra that contains $\mathcal{E}$ so the intersection is non-empty. By \cref{capofSigmaAisSigmaA}, $\sigma(\mathcal{E})$ is a $\sigma$-algebra.
\end{proof}

Similar to the above, $\delta(\mathcal{E})$ is defined as the $\lambda$-system generated by $\mathcal{E}$.

\vspace{2mm}
We always have the following:
\begin{enumerate}
    \item $\mathcal{E}\subseteq \sigma(\mathcal{E})$.
    \item If $\mathcal{E}_1\subseteq\mathcal{E}_2$, then $\sigma(\mathcal{E}_1)\subseteq\sigma(\mathcal{E}_2)$.
    \item $\mathcal{A}$ is a $\sigma$-algebra if and only if $\sigma(\mathcal{A})=\mathcal{A}$.
\end{enumerate}
Similar statements hold for $\lambda$-systems. Further, $\delta(\mathcal{E})\subseteq\sigma(\mathcal{E})$. This is to be expected as $\sigma$-algebras have more ``structure" than $\lambda$-systems.

\begin{theorem}[$\cap$-closed $\lambda$-system]
\label{cap closed lam sys}
    Let $\mathcal{D}\subseteq2^\Omega$ be a $\lambda$-system. Then $\mathcal{D}$ is a $\pi$-system if and only if $\mathcal{D}$ is a $\sigma$-algebra.
\end{theorem}
\begin{proof}
    If $\mathcal{D}$ is a $\sigma$-algebra, then it is obviously a $\pi$-system. Let $\mathcal{D}$ be a $\pi$-system. Then
    \begin{enumerate}[(a)]
        \item As $\mathcal{D}$ is a $\lambda$-system, $\Omega\in\mathcal{D}$.
        \item Let $A\in\mathbb{D}$. Since $\Omega\in\mathcal{D}$ and $\mathcal{D}$ is a $\lambda$-system, $A^c=\Omega\setminus A\in\mathcal{D}$.
        \item Let $A,B\in\mathcal{D}$. We have $A\cap B\in\mathcal{D}$. We now have $A\setminus B = A\setminus (A\cap B)\in\mathcal{D}$, that is, $\mathcal{D}$ is $\setminus$-closed.
        
        Let $A_1,A_2,\ldots\in\mathcal{D}$. Then by \cref{if setminus closed}, there exist $B_1,B_2,\ldots\in\mathcal{D}$ such that
        $$\bigcup_{i=1}^\infty A_i=\biguplus_{i=1}^\infty B_i\in\mathcal{D}.$$
    \end{enumerate}
    This completes the proof.
\end{proof}

\begin{theorem}[Dynkin's $\pi$-$\lambda$ theorem]
\label{dynkins pi lam theorem}
    If $\mathcal{E}\subseteq2^\Omega$ is a $\pi$-system, then $\delta(\mathcal{E})=\sigma(\mathcal{E})$.
\end{theorem}
\begin{proof}
    We already have $\delta(\mathcal{E})\subseteq\sigma(\mathcal{E})$. We must now prove the reverse inclusion. We shall show that $\delta(\mathcal{E})$ is a $\pi$-system.
    
    For each $E\in\delta(\mathcal{E})$, let
    $$\mathcal{D}_E=\{A\in\delta(\mathcal{E}) : A\cap E\in\delta(\mathcal{E})\}.$$
    
    To show that $\delta(\mathcal{E})$ is a $\pi$-system, it suffices to show that $\delta(\mathcal{E})\subseteq\mathcal{D}_E$ for all $E\in\delta(\mathcal{E})$. We shall first show that $\mathcal{D}_E$ is a $\lambda$-system for each $E\in\mathcal{E}$ by checking each of the conditions in \cref{defLamSystem}.
    \begin{enumerate}[(a)]
        \item We clearly have $\Omega\in\mathcal{D}_E$ as $\Omega\cap E=E$.
        \item For any $A,B\in\mathcal{D}_E$ with $A\subseteq B$, 
        $$(B\setminus A)\cap E = (B\cap E)\setminus(A\cap E)\in\delta(\mathcal{E}).$$
        \item Let $A_1,A_2,\ldots\in\mathcal{D}_E$ be mutually disjoint sets. Then
        $$\left(\biguplus_{i=1}^\infty A_i\right)\cap E = \biguplus_{i=1}^\infty\left(A_i\cap E\right)\in\delta(\mathcal{E}).$$
    \end{enumerate}
    Now since $\mathcal{D}_E$ is a $\lambda$-system and $\mathcal{E}\subseteq\mathcal{D}_E$ (Why?), $\delta(\mathcal{E})\subseteq\mathcal{D}_E$.
    
    Now that we have shown that $\delta(\mathcal{E})$ is a $\pi$-system, the result follows by \cref{cap closed lam sys}.
\end{proof}

\begin{definition}[Topology]
    Let $\Omega\neq\emptyset$ be an arbitrary set. A class of sets $\tau\subseteq2^\Omega$ is called a \textit{topology} on $2^\Omega$ if
    \begin{enumerate}[(i)]
        \item $\emptyset,\Omega\in\tau$,
        \item $\tau$ is $\cap$-closed, and
        \item for any $\mathcal{F}\subseteq\tau$, $\bigcup_{A\in\mathcal{F}}A\in\tau.$
    \end{enumerate}
\end{definition}

In the above case, the pair $(\Omega,\tau)$ is called a \textit{topological space}. The sets $A\in\tau$ are called \textit{open} and the sets $A\subseteq\Omega$ with $A^c\in\tau$ are called \textit{closed}.

\vspace{1mm}
Note that in contrast with $\sigma$-algebras, topologies are closed under only finite intersections but are also closed under arbitrary unions.

\vspace{2mm}
For example, consider the natural topology on $\mathbb{R}$ which consists of all open intervals in $\mathbb{R}$ and any arbitrary union of them.

\begin{definition}[Borel $\sigma$-algebra]
    Let $(\Omega,\tau)$ be a topological space. The $\sigma$-algebra
    $$\mathcal{B}(\Omega)=\mathcal{B}(\Omega,\tau)=\sigma(\tau)$$
    that is generated by the open sets is called the \textit{Borel $\sigma$-algebra on $\Omega$}. The elements $A\in\mathcal{B}(\Omega,\tau)$ are called \textit{Borel sets} or \textit{Borel measurable sets}.
\end{definition}


A Borel $\sigma$-algebra that we shall often encounter is $\mathcal{B}(\mathbb{R}^n)$ for $n\in\mathbb{N}$. Consider the following classes of sets:
\begin{align*}
    \mathcal{A}_1 &= \{A\subseteq\mathbb{R}^n:A\text{ is open}\} \\
    \mathcal{A}_2 &= \{A\subseteq\mathbb{R}^n:A\text{ is closed}\} \\
    \mathcal{A}_3 &= \{A\subseteq\mathbb{R}^n:A\text{ is compact}\} \\
    \mathcal{A}_4 &= \{(a,b):a,b\in\mathbb{Q}^n\text{ and }a<b\} \\
    \mathcal{A}_5 &= \{(a,b]:a,b\in\mathbb{Q}^n\text{ and }a<b\} \\
    \mathcal{A}_6 &= \{[a,b):a,b\in\mathbb{Q}^n\text{ and }a<b\} \\
    \mathcal{A}_7 &= \{[a,b]:a,b\in\mathbb{Q}^n\text{ and }a<b\} \\
    \mathcal{A}_8 &= \{(-\infty,b):b\in\mathbb{Q}^n\} \\
    \mathcal{A}_9 &= \{(-\infty,b]:b\in\mathbb{Q}^n\} \\
    \mathcal{A}_{10} &= \{(a,\infty):a\in\mathbb{Q}^n\} \\
    \mathcal{A}_{11} &= \{[a,\infty):a\in\mathbb{Q}^n\} \\
\end{align*}
It may be proved that $\mathcal{B}(\mathbb{R}^n)$ is generated by any of the classes of sets $\mathcal{A}_1,\mathcal{A}_2,\ldots,\mathcal{A}_{11}$.

\vspace{2mm}
For $A\in\mathcal{B}(\mathbb{R})$, we represent by $\left.\mathcal{B}(\mathbb{R})\right|_{A}$ the restriction of $\mathcal{B}(\mathbb{R})$ to $A$. It may be proved that this is equal to $\mathcal{B}(A)$, the $\sigma$-algebra generated by the open subsets of $A$.

\subsection{Measure}

\begin{definition}
    Let $\mathcal{A}\subseteq2^\Omega$ and let $\mu:\mathcal{A}\to[0,\infty]$ be a set function. We say that $\mu$ is
    \begin{enumerate}[(i)]
        \item \textit{monotone} if for any $A,B\in\mathcal{A}$, $A\subseteq B$ implies that $\mu(A)\leq\mu(B)$,
        
        \item \textit{additive} if for any choice of finitely many mutually disjoint sets $A_1,\ldots,A_n\in\mathcal{A}$ with $\biguplus_{i=1}^n A_i\in\mathcal{A}$,
        $$\mu\left(\biguplus_{i=1}^nA_i\right)=\sum_{i=1}^n\mu(A_i),$$
        
        \item $\sigma$-additive if for any choice of countably many mutually disjoint sets $A_1,A_2,\ldots\in\mathcal{A}$ with $\biguplus_{i=1}^\infty A_i\in\mathcal{A}$,
        $$\mu\left(\biguplus_{i=1}^\infty A_i\right)=\sum_{i=1}^\infty\mu(A_i),$$
        
        \item subadditive if for any choice of finitely many sets $A,A_1,A_2,\ldots,A_n\in\mathcal{A}$ with $A\subseteq\bigcup_{i=1}^nA_i$, we have $$\mu(A)\leq\sum_{i=1}^n\mu(A_i),\text{ and}$$
        
        \item $\sigma$-subadditive if for any choice of countably many sets $A,A_1,A_2,\ldots\in\mathcal{A}$ with $A\subseteq\bigcup_{i=1}^\infty A_i$, we have
        $$\mu(A)\leq\sum_{i=1}^\infty \mu(A_i).$$
    \end{enumerate}
\end{definition}

\begin{definition}
\label{measureDef}
    Let $\mathcal{A}$ be a semiring and $\mu:\mathcal{A}\to[0,\infty]$ be a set function with $\mu(\emptyset)=0$. $\mu$ is called a
    \begin{enumerate}[(i)]
        \item \textit{content} if $\mu$ is additive,
        \item \textit{premeasure} if $\mu$ is $\sigma$-additive, and
        \item \textit{measure} if $\mu$ is $\sigma$-additive and $\mathcal{A}$ is a $\sigma$-algebra.
    \end{enumerate}
\end{definition}

\begin{theorem}[Properties of contents]
\label{properties of content}
    Let $\mathcal{A}$ be a semiring and $\mu$ be a content on $\mathcal{A}$. Then
    \begin{enumerate}[(a)]
        \item If $\mathcal{A}$ is a ring, then $\mu(A\cup B)+\mu(A\cap B)=\mu(A)+\mu(B)$ for any $A,B\in\mathcal{A}$.
        
        \item $\mu$ is monotone. If $\mathcal{A}$ is a ring, then $\mu(B)=\mu(A)+\mu(B\setminus A)$ for any $A,B\in\mathcal{A}$ with $A\subseteq B$.
        
        \item $\mu$ is subadditive. If $\mu$ is $\sigma$-additive, then it is also $\sigma$-subadditive.
        
        \item If $\mathcal{A}$ is a ring, then $$\sum_{n=1}^\infty \mu(A_n)\leq\mu\left(\bigcup_{n=1}^\infty A_n\right)$$ for any choice of countably many mutually disjoint sets $A_1,A_2,\ldots\in\mathcal{A}$ with $\bigcup_{i=1}^\infty A_i\in\mathcal{A}$.
    \end{enumerate}
\end{theorem}
\begin{proof}
~
    \begin{enumerate}[(a)]
        \item Note that $A\cup B=A\uplus (B\setminus A)$ and $B=(A\cap B)\uplus (B\setminus A)$. As $\mu$ is additive,
        $$\mu(A\cup B)=\mu(A)+\mu(B\setminus A)\text{ and }\mu(B)=\mu(A\cap B)+\mu(B\setminus A).$$
        The result follows.
        
        \item Let $A\subseteq B$. If $B\setminus A\in\mathcal{A}$ (which is true in the case of a ring), we have $B=A\uplus (B\setminus A)$ and thus
        $$\mu(B)=\mu(A)+\mu(B\setminus A).$$
        If $\mathcal{A}$ is just a semiring, then there exist $n\in\mathbb{N}$ and mutually disjoint sets $C_1,C_2,\ldots,C_n\in\mathcal{A}$ such that $$B\setminus A=\biguplus_{i=1}^n C_i.$$
        In either case, we have $\mu(A)\leq \mu(B)$.
        
        \item Let $A, A_1,A_2,\ldots,A_n\in\mathcal{A}$ such that $A\subseteq\bigcup_{i=1}^n A_i$. Let $B_1=A_1$ and for each $k=2,3,\ldots,n$, let
        $$B_k = A_k\setminus\left(\bigcup_{i=1}^{k-1}A_i\right).$$
        Note that any two $B_i$s are disjoint. As $\mu$ is additive and monotone, we have
        \begin{align*}
            \mu(A) &\leq \mu\left(\bigcup_{i=1}^n A_i\right) \\
            &= \mu\left(\biguplus_{i=1}^n B_i\right) \\
            &= \sum_{i=1}^n \mu(B_i) \leq \sum_{i=1}^n \mu(A_i).
        \end{align*}
        We can similarly prove that if $\mu$ is $\sigma$-additive, then it is $\sigma$-subadditive.
        
        \item Let $A=\bigcup_{i=1}^\infty A_i\in\mathcal{A}$. Since $\mu$ is additive and monotone,
        $$\sum_{i=1}^m\mu(A_i)=\mu\left(\biguplus_{i=1}^m A_i\right)\leq \mu(A)\text{ for any $m\in\mathbb{N}$.}$$
        The result follows.
    \end{enumerate}
\end{proof}

Note that if equality holds in the fourth part of the above theorem, $\mu$ is a premeasure.

\begin{definition}[Finite content]
    Let $\mathcal{A}$ be a semiring. A content $\mu$ on $A$ is called
    \begin{enumerate}[(i)]
        \item \textit{finite} if $\mu(A)<\infty$ for all $A\in\mathcal{A}$ and
        \item \textit{$\sigma$-finite} if there exists a sequence of sets $\Omega_1,\Omega_2,\ldots\in\mathcal{A}$ such that $\Omega=\bigcup_{i=1}^\infty\Omega_i$ and $\mu(\Omega_i)<\infty$ for every $i\in\mathbb{N}$.
    \end{enumerate}
\end{definition}

\begin{definition}
    Let $A,A_1,A_2,\ldots$ be sets. We write
    \begin{enumerate}[(i)]
        \item $A_n\uparrow A$ if $A_1\subseteq A_2\subseteq A_3\subseteq\cdots$ and $\bigcup_{i=1}^\infty A_i=A$. In this case, we say that $A_n$ increases to $A$.
        \item $A_n\downarrow A$ if $A_1\supseteq A_2\supseteq A_3\supseteq\cdots$ and $\bigcap_{i=1}^\infty A_i=A$. In this case, we say that $A_n$ decreases to $A$.
    \end{enumerate}
\end{definition}

For example, if $A_n=\left(-\frac{1}{n},\frac{1}{n}\right)$ for $n\in\mathbb{N}$, then $A_n\downarrow \{0\}$.

\begin{definition}[Continuity of contents]
    Let $\mu$ be a content on the ring $\mathcal{A}$. $\mu$ is called
    \begin{enumerate}[(i)]
        \item \textit{lower semicontinuous} if $\lim_{n\to\infty}\mu(A_n)=\mu(A)$ for any $A\in\mathcal{A}$ and sequence $(A_n)_{n\in\mathbb{N}}$ such that $A_n\uparrow A$,
        \item \textit{upper semicontinuous} if $\lim_{n\to\infty}\mu(A_n)=\mu(A)$ for any $A\in\mathcal{A}$ and sequence $(A_n)_{n\in\mathbb{N}}$ such that $\mu(A_n)<\infty$ for some $n$ (this implies that it holds for all $n\in\mathbb{N}$) and $A_n\downarrow A$,
        \item \textit{$\emptyset$-continuous} if (ii) holds for $A=\emptyset$.
    \end{enumerate}
\end{definition}

\begin{theorem}
\label{tripledoubleEquivalence}
    Let $\mu$ be a content on the ring $\mathcal{A}$. The following properties are equivalent:
    \begin{enumerate}[(a)]
        \item $\mu$ is $\sigma$-additive (and hence a premeasure).
        \item $\mu$ is $\sigma$-subadditive.
        \item $\mu$ is lower semicontinuous.
        \item $\mu$ is $\emptyset$-continuous.
        \item $\mu$ is upper semicontinuous.
    \end{enumerate}
    Then (a)$\iff$(b)$\iff$(c)$\implies$(d)$\iff$(e).
    If $\mu$ is finite, then all five statements are equivalent.
\end{theorem}
\begin{proof}
~
    \begin{itemize}
        \item (a)$\implies$(b) ($\sigma$-additivity implies $\sigma$-subadditivity).
        
        This follows from \cref{properties of content}(c).
        
        \item (b)$\implies$(a) ($\sigma$-subadditivity implies $\sigma$-additivity).
        
        This follows from \cref{properties of content}(d).
        
        \item (a)$\implies$(c) ($\sigma$-additivity implies lower semicontinuity).
        
        Let $\mu$ be a premeasure and $A\in\mathcal{A}$. Let $A_1,A_2,\ldots\in\mathcal{A}$ such that $A_n\uparrow A$ and let $A_0=\emptyset$. Then
        $$\mu(A)=\sum_{i=1}^\infty \mu(A_i\setminus A_{i-1})=\lim_{n\to\infty}\sum_{i=1}^n\mu(A_i\setminus A_{i-1})=\lim_{n\to\infty}\mu(A_n).$$
        
        \item (c)$\implies$(a) (lower semicontinuity implies $\sigma$-additivity).
        
        Let $B_1,B_2,\ldots\in\mathcal{A}$ be mutually disjoint and let $B=\biguplus_{n=1}^\infty B_n\in\mathcal{A}$. Let $A_n=\biguplus_{i=1}^n B_i$ for each $n\in\mathbb{N}$. Then
        $$\mu(B)=\lim_{n\to\infty}\mu(A_n)=\sum_{i=1}^\infty\mu(B_i).$$
        Thus $\mu$ is $\sigma$-additive.
        
        \item (d)$\implies$(e) ($\emptyset$-continuity implies upper semicontinuity).
        
        Let $A,A_1,A_2,\ldots\in\mathcal{A}$ with $A_n\downarrow A$ and $\mu(A_1)<\infty$. Define $B_n=A_n\setminus A\in\mathcal{A}$ for valid $n$. Then $B_n\downarrow\emptyset$. Thus
        $$\lim_{n\to\infty}\mu(A_n)-\mu(A)=\lim_{n\to\infty}\mu(B_n)=0$$
        and the result is proved.
        
        \item (e)$\implies$(d) (upper semicontinuity implies $\emptyset$-continuity).
        
        This is obvious.
        
        \item (c)$\implies$(d) (lower semicontinuity implies $\emptyset$-continuity).
        
        Let $A_1,A_2,\ldots\in\mathcal{A}$ with $A_n\downarrow\emptyset$ and $\mu(A_1)<\infty$. Then $A_1\setminus A_n\in\mathcal{A}$ for all $n\in\mathbb{N}$ and $A_1\setminus A_n\uparrow A_1$. Thus
        $$\mu(A_1)=\lim_{n\to\infty}\mu(A_1)-\mu(A_n).$$
        Since $\mu(A_1)<\infty$, $\lim_{n\to\infty}A_n=0$ and the result is proved.
        
        \item (d)$\implies$(c) ($\emptyset$-continuity implies lower semicontinuity) if $\mu$ is finite.
        
        Let $A,A_1,A_2,\ldots\in\mathcal{A}$ with $A_n\uparrow A$. Then $A\setminus A_n\downarrow\emptyset$ and
        $$\lim_{n\to\infty}\mu(A)-\mu(A_n)=\lim_{n\to\infty}\mu(A\setminus A_n)=0.$$
        The result follows.
    \end{itemize}
\end{proof}

\begin{definition}[Measurable spaces]
~
\begin{enumerate}[(i)]
    \item A pair $(\Omega,\mathcal{A})$ consisting of a nonempty set $\Omega$ and a $\sigma$-algebra $\mathcal{A}\subseteq2^\Omega$ is called a \textit{measurable space}. The sets $A\in\mathcal{A}$ are called \textit{measurable sets}. If $\Omega$ is countable and $\mathcal{A}=2^\Omega$, then the space $(\Omega,2^\Omega)$ is called \textit{discrete}.
    
    \item A triple $(\Omega,\mathcal{A},\mu)$ is called a \textit{measure space} if $(\Omega,\mathcal{A})$ is a measurable space and $\mu$ is a measure on $\mathcal{A}$.
\end{enumerate}
\end{definition}


\subsection{Measurable Maps}

In measure theory, the homomorphisms (structure-preserving maps between objects) are studied as measurable maps.

\begin{definition}[Measurable map]
    Let $(\Omega,\mathcal{A})$ and $(\Omega',\mathcal{A}')$ be measurable spaces. A map $X:\Omega\to\Omega'$ is called \textit{$\mathcal{A}-\mathcal{A}'$-measurable} (or just measurable) if
    $$X^{-1}(A')\in\mathcal{A}\text{ for any }A'\in\mathcal{A}'.$$
\end{definition}

In this case, we write $X:(\Omega,\mathcal{A})\to(\Omega',\mathcal{A}')$.

\begin{theorem}[Generated $\sigma$-algebra]
    Let $(\Omega',\mathcal{A}')$ be a measurable space and $\Omega$ be a nonempty set. Let $X:\Omega\to\Omega'$ be a map. Then
    $$X^{-1}(\mathcal{A}')=\{X^{-1}(A'):A'\in\mathcal{A}'\}$$
    is the smallest $\sigma$-algebra with respect to which $X$ is measurable. We call $X^{-1}(\mathcal{A}')$ the \textit{$\sigma$-algebra generated by $X$} and denote it as $\sigma(X)$.
\end{theorem}
\begin{proof}
    Let $X$ be measurable with respect to some $\sigma$-algebra $\mathcal{A}$. Then $X^{-1}(A')\in\mathcal{A}$ for any $A'\in\mathcal{A}'$, that is, $\sigma(X)\subseteq\mathcal{A}$. Let us now prove that $\sigma(X)$ is a $\sigma$-algebra by checking each of the axioms in \cref{defSigAlgebra}.
    \begin{enumerate}
        \item As $\Omega'\in\mathcal{A}'$ and $X^{-1}(\Omega')=\Omega$, $\Omega\in\sigma(X)$.
        
        \item Let $A\in\sigma(X)$ and $A'\in\mathcal{A}'$ such that $X^{-1}(A')=A$. Then as $\mathcal{A}'$ is closed under complements,
        $$\Omega\setminus A = X^{-1}(\Omega')\setminus X^{-1}(A') = X^{-1}(\Omega'\setminus A')\in\sigma(X).$$
        Therefore, $\sigma(X)$ is closed under complements.
        
        \item Let $A_1,A_2\ldots\in\sigma(X)$ and $A_1',A_2',\ldots\in\mathcal{A}'$ such that $A_i=X^{-1}(A_i')$ for each $i\in\mathbb{N}$. Then as $\mathcal{A}'$ is $\sigma$-$\cup$-closed,
        $$\bigcup_{i\in\mathbb{N}} A_i = \bigcup_{i\in\mathbb{N}} X^{-1} (A_i') = X^{-1}\left(\bigcup_{i\in\mathbb{N}} A_i'\right) \in\sigma(X)$$
    \end{enumerate}
    Therefore, $\sigma(X)$ is a $\sigma$-algebra.
\end{proof}

\begin{theorem}
\label{generating pi system fixed under preimage}
    Let $(\Omega,\mathcal{A})$ and $(\Omega',\mathcal{A}')$ be measurable spaces and $X:\Omega\to\Omega'$ be a map.
    Let $\mathcal{E}'\subseteq\mathcal{A}'$ be a class of sets. Then $\sigma(X^{-1}(\mathcal{E}'))=X^{-1}(\sigma(\mathcal{E}'))$.
\end{theorem}
\begin{proof}
    We have that $X^{-1}(\mathcal{E})\subseteq X^{-1}(\sigma(\mathcal{E}))=\sigma(X^{-1}(\sigma(\mathcal{E})))$. This implies that
    $$\sigma(X^{-1}(\mathcal{E}))\subseteq X^{-1}(\sigma(\mathcal{E})).$$
    To establish the reverse inclusion, consider
    $$\mathcal{A}_0'=\{A'\in\sigma(\mathcal{E}'):X^{-1}(A')\in\sigma(X^{-1}(\mathcal{E}'))\}$$
    We shall show that $\mathcal{A}_0'$ is a $\sigma$-algebra.
    \begin{enumerate}[(a)]
        \item Clearly, $\Omega'\in\mathcal{A}_0'$ as $\Omega\in\sigma(X^{-1}(\mathcal{E}'))$ and $\Omega'\in\sigma(\mathcal{E}')$.
        
        \item Let $A_0'\in\mathcal{A}_0'$. Then
        $$X^{-1}((A_0')^c) = (X^{-1}(A_0'))^c \in \sigma(X^{-1}(\mathcal{E}'))$$
        and thus $\mathcal{A_0}'$ is closed under complements.
        
        \item Let $A_1',A_2',\ldots\in\mathcal{A}_0'$. Then
        $$X^{-1}\left(\bigcup_{i=1}^\infty A_i'\right) = \bigcup_{i=1}^\infty X^{-1}\left(A_i'\right)\in \sigma(X^{-1}(\mathcal{E}')).$$
        Thus, $\mathcal{A}_0'$ is $\sigma$-$\cup$-closed.
    \end{enumerate}
    
    Now, note that $\mathcal{E}'\subseteq\mathcal{A}_0'$ and $\mathcal{A}_0'\subseteq\sigma(\mathcal{E}')$. This implies that $\mathcal{A}_0'=\sigma(\mathcal{E}')$, and thus $X^{-1}(\sigma(\mathcal{E}'))\subseteq\sigma(X^{-1}(\mathcal{E}'))$.
    
    This proves the result.
\end{proof}

\begin{corollary}
\label{measurable if generator inv}
    Let $(\Omega,\mathcal{A})$ and $(\Omega',\mathcal{A}')$ be measurable spaces and $X:\Omega\to\Omega'$ be a map. Let $\mathcal{E}'\subseteq\mathcal{A}'$ be a class of sets. Then $X$ is $\mathcal{A}$-$\sigma(\mathcal{E}')$ measurable if and only if $X^{-1}(\mathcal{E}')\in\mathcal{A}$. If in particular $\sigma(\mathcal{E}')=\mathcal{A}'$, then $X$ is $\mathcal{A}-\mathcal{A}'$-measurable if and only if $X^{-1}(\mathcal{E}')\subseteq\mathcal{A}'$.
\end{corollary}

\begin{theorem}
        Let $(\Omega,\mathcal{A})$, $(\Omega',\mathcal{A}')$, and $(\Omega'',\mathcal{A}'')$ be measurable spaces and let $X:\Omega\to\Omega'$ and $X':\Omega'\to\Omega''$ be measurable. Then $Y=X'\circ X:\Omega\to\Omega''$ is $\mathcal{A}-\mathcal{A}''$-measurable.
\end{theorem}
\begin{proof}
    This is due to the fact that
    $$Y^{-1}(\mathcal{A}'')\subseteq X^{-1}((X^{-1})(\mathcal{A}''))\subseteq X^{-1}(\mathcal{A}')\subseteq \mathcal{A}.$$
\end{proof}

The above theorem just states that the composition of two measurable maps is measurable.

\begin{theorem}[Measurability of Continuous Maps]
    Let $(\Omega,\tau)$ and $(\Omega',\tau')$ be topological spaces and let $f:\Omega\to\Omega'$ be a continuous map. Then $f$ is $\mathcal{B}(\Omega)-\mathcal{B}(\Omega')$-measurable.
\end{theorem}
\begin{proof}
    As $\mathcal{B}(\Omega')=\sigma(\tau')$, by \cref{measurable if generator inv} it is enough to show that $f^{-1}(A')\in\sigma(\tau)$ for all $A'\in\tau'$. However, since $f$ is continuous, $f^{-1}(A')\in\tau$ for all $A'\in\tau'$ so the result follows.
\end{proof}

\begin{theorem}
    Let $X_1,X_2,\ldots$ be measurable maps $(\Omega,\mathcal{A})\to(\overline{\mathbb{R}}, \mathcal{B}(\overline{\mathbb{R}}))$. Then $\inf_{n\in\mathbb{N}}X_n$, $\sup_{n\in\mathbb{N}}X_n$, $\liminf_{n\in\mathbb{N}}X_n$, and $\limsup_{n\in\mathbb{N}}X_n$ are also measurable.
\end{theorem}
\begin{proof}
    For any $x\in\overline{\mathbb{R}}$,
    $$\left(\inf_{n\in\mathbb{N}}X_n\right)^{-1}([-\infty,x))=\bigcup_{n=1}^\infty (X_n)^{-1}([-\infty,x))\in\mathcal{A}.$$
    The first part of the result follows by \cref{measurable if generator inv}. The proof for $\sup_{n\in\mathbb{N}}X_n$ is similar.
    
    For $\limsup_{n\in\mathbb{N}}$, consider the sequence $(Y_n)_{n\in\mathbb{N}}$ where $Y_n=\sup_{m\geq n}X_m$. Each $Y_m$ is measurable. Then since $\inf_{n\in\mathbb{N}}Y_m$ is measurable, the result follows.
\end{proof}

\begin{definition}[Simple Function]
    Let $(\Omega,\mathcal{A})$ be a measurable space. A map $f:\Omega\to\mathbb{R}$ is called \textit{simple} if there exists some $n\in\mathbb{N}$, mutually disjoint sets $A_1,\ldots,A_n\in\mathcal{A}$, and $\alpha_1,\ldots,\alpha_n\in\mathbb{R}$ such that
    $$f=\sum_{i=1}^n \alpha_i\indic_{A_i}.$$
\end{definition}

\begin{definition}
    Let $f,f_1,f_2,\ldots$ be maps $\Omega\to\overline{\mathbb{R}}$ such that
    $$f_1(\omega)\leq f_2(\omega)\leq\cdots\text{ and }\lim_{n\to\infty}f_n(\omega)=\omega\text{ for all }\omega\in\Omega.$$
    We then write $f_n\uparrow f$. Similarly, we write $f_n\downarrow f$ if $(-f_n)\uparrow(-f)$.
\end{definition}

\begin{theorem}
\label{measurable function simple sequence}
    Let $(\Omega,\mathcal{A})$ be a measurable space and let $f:\Omega\to[0,\infty]$ be measurable. Then
    \begin{enumerate}[(a)]
        \item There exists a sequence $(f_n)_{n\in\mathbb{N}}$ of non-negative simple functions such that $f_n\uparrow f$.
        \item There are measurable sets $A_1,A_2,\ldots\in\mathcal{A}$ and $\alpha_1,\alpha_2,\ldots\in[0,\infty)$ such that $f=\sum_{i=1}^n \alpha_i\indic_{A_i}$.
    \end{enumerate}
\end{theorem}
\begin{proof}
    ~
    \begin{enumerate}[(a)]
        \item For $n\in\mathbb{N}_0$, define
        $$f_n=\min\{n, 2^{-n}\lfloor2^nf\rfloor\}.$$
        Each $f_n$ is measurable. (Why?) Since it can take at most $n2^n+1$ distinct values, each $f_n$ is simple. Clearly, $f_n\uparrow f$.
        
        \item Let $f_n$ be the same as above. For $n\in\mathbb{N}$ and $i\in[2^n]$, define
        $$B_{n,i}=\{\omega:f_n(\omega)-f_{n-1}(\omega)=i2^{-n}\}\text{ and }\beta_{n,i}=i2^{-n}.$$
        Then $f_n-f_{n-1}=\sum_{i=1}^{2^n} \beta_{n,i}\indic_{B_{n,i}}$. Changing the enumeration from $(n,i)$ to $m$, we get some $(\alpha_m)_{m\in\mathbb{N}}$ and $(A_m)_{m\in\mathbb{N}}$ such that
        $$f=f_0 + \sum_{n=1}^\infty(f_n-f_{n-1})=\sum_{m=1}^\infty \alpha_m\indic_{A_m}.$$
    \end{enumerate}
\end{proof}

\clearpage